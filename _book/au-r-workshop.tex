\documentclass[]{book}
\usepackage{lmodern}
\usepackage{amssymb,amsmath}
\usepackage{ifxetex,ifluatex}
\usepackage{fixltx2e} % provides \textsubscript
\ifnum 0\ifxetex 1\fi\ifluatex 1\fi=0 % if pdftex
  \usepackage[T1]{fontenc}
  \usepackage[utf8]{inputenc}
\else % if luatex or xelatex
  \ifxetex
    \usepackage{mathspec}
  \else
    \usepackage{fontspec}
  \fi
  \defaultfontfeatures{Ligatures=TeX,Scale=MatchLowercase}
\fi
% use upquote if available, for straight quotes in verbatim environments
\IfFileExists{upquote.sty}{\usepackage{upquote}}{}
% use microtype if available
\IfFileExists{microtype.sty}{%
\usepackage{microtype}
\UseMicrotypeSet[protrusion]{basicmath} % disable protrusion for tt fonts
}{}
\usepackage[margin=1in]{geometry}
\usepackage{hyperref}
\hypersetup{unicode=true,
            pdftitle={Introduction to R for Natural Resource Scientists},
            pdfauthor={Ben Staton},
            pdfborder={0 0 0},
            breaklinks=true}
\urlstyle{same}  % don't use monospace font for urls
\usepackage{natbib}
\bibliographystyle{apalike}
\usepackage{color}
\usepackage{fancyvrb}
\newcommand{\VerbBar}{|}
\newcommand{\VERB}{\Verb[commandchars=\\\{\}]}
\DefineVerbatimEnvironment{Highlighting}{Verbatim}{commandchars=\\\{\}}
% Add ',fontsize=\small' for more characters per line
\usepackage{framed}
\definecolor{shadecolor}{RGB}{248,248,248}
\newenvironment{Shaded}{\begin{snugshade}}{\end{snugshade}}
\newcommand{\KeywordTok}[1]{\textcolor[rgb]{0.13,0.29,0.53}{\textbf{#1}}}
\newcommand{\DataTypeTok}[1]{\textcolor[rgb]{0.13,0.29,0.53}{#1}}
\newcommand{\DecValTok}[1]{\textcolor[rgb]{0.00,0.00,0.81}{#1}}
\newcommand{\BaseNTok}[1]{\textcolor[rgb]{0.00,0.00,0.81}{#1}}
\newcommand{\FloatTok}[1]{\textcolor[rgb]{0.00,0.00,0.81}{#1}}
\newcommand{\ConstantTok}[1]{\textcolor[rgb]{0.00,0.00,0.00}{#1}}
\newcommand{\CharTok}[1]{\textcolor[rgb]{0.31,0.60,0.02}{#1}}
\newcommand{\SpecialCharTok}[1]{\textcolor[rgb]{0.00,0.00,0.00}{#1}}
\newcommand{\StringTok}[1]{\textcolor[rgb]{0.31,0.60,0.02}{#1}}
\newcommand{\VerbatimStringTok}[1]{\textcolor[rgb]{0.31,0.60,0.02}{#1}}
\newcommand{\SpecialStringTok}[1]{\textcolor[rgb]{0.31,0.60,0.02}{#1}}
\newcommand{\ImportTok}[1]{#1}
\newcommand{\CommentTok}[1]{\textcolor[rgb]{0.56,0.35,0.01}{\textit{#1}}}
\newcommand{\DocumentationTok}[1]{\textcolor[rgb]{0.56,0.35,0.01}{\textbf{\textit{#1}}}}
\newcommand{\AnnotationTok}[1]{\textcolor[rgb]{0.56,0.35,0.01}{\textbf{\textit{#1}}}}
\newcommand{\CommentVarTok}[1]{\textcolor[rgb]{0.56,0.35,0.01}{\textbf{\textit{#1}}}}
\newcommand{\OtherTok}[1]{\textcolor[rgb]{0.56,0.35,0.01}{#1}}
\newcommand{\FunctionTok}[1]{\textcolor[rgb]{0.00,0.00,0.00}{#1}}
\newcommand{\VariableTok}[1]{\textcolor[rgb]{0.00,0.00,0.00}{#1}}
\newcommand{\ControlFlowTok}[1]{\textcolor[rgb]{0.13,0.29,0.53}{\textbf{#1}}}
\newcommand{\OperatorTok}[1]{\textcolor[rgb]{0.81,0.36,0.00}{\textbf{#1}}}
\newcommand{\BuiltInTok}[1]{#1}
\newcommand{\ExtensionTok}[1]{#1}
\newcommand{\PreprocessorTok}[1]{\textcolor[rgb]{0.56,0.35,0.01}{\textit{#1}}}
\newcommand{\AttributeTok}[1]{\textcolor[rgb]{0.77,0.63,0.00}{#1}}
\newcommand{\RegionMarkerTok}[1]{#1}
\newcommand{\InformationTok}[1]{\textcolor[rgb]{0.56,0.35,0.01}{\textbf{\textit{#1}}}}
\newcommand{\WarningTok}[1]{\textcolor[rgb]{0.56,0.35,0.01}{\textbf{\textit{#1}}}}
\newcommand{\AlertTok}[1]{\textcolor[rgb]{0.94,0.16,0.16}{#1}}
\newcommand{\ErrorTok}[1]{\textcolor[rgb]{0.64,0.00,0.00}{\textbf{#1}}}
\newcommand{\NormalTok}[1]{#1}
\usepackage{longtable,booktabs}
\usepackage{graphicx,grffile}
\makeatletter
\def\maxwidth{\ifdim\Gin@nat@width>\linewidth\linewidth\else\Gin@nat@width\fi}
\def\maxheight{\ifdim\Gin@nat@height>\textheight\textheight\else\Gin@nat@height\fi}
\makeatother
% Scale images if necessary, so that they will not overflow the page
% margins by default, and it is still possible to overwrite the defaults
% using explicit options in \includegraphics[width, height, ...]{}
\setkeys{Gin}{width=\maxwidth,height=\maxheight,keepaspectratio}
\IfFileExists{parskip.sty}{%
\usepackage{parskip}
}{% else
\setlength{\parindent}{0pt}
\setlength{\parskip}{6pt plus 2pt minus 1pt}
}
\setlength{\emergencystretch}{3em}  % prevent overfull lines
\providecommand{\tightlist}{%
  \setlength{\itemsep}{0pt}\setlength{\parskip}{0pt}}
\setcounter{secnumdepth}{5}
% Redefines (sub)paragraphs to behave more like sections
\ifx\paragraph\undefined\else
\let\oldparagraph\paragraph
\renewcommand{\paragraph}[1]{\oldparagraph{#1}\mbox{}}
\fi
\ifx\subparagraph\undefined\else
\let\oldsubparagraph\subparagraph
\renewcommand{\subparagraph}[1]{\oldsubparagraph{#1}\mbox{}}
\fi

%%% Use protect on footnotes to avoid problems with footnotes in titles
\let\rmarkdownfootnote\footnote%
\def\footnote{\protect\rmarkdownfootnote}

%%% Change title format to be more compact
\usepackage{titling}

% Create subtitle command for use in maketitle
\newcommand{\subtitle}[1]{
  \posttitle{
    \begin{center}\large#1\end{center}
    }
}

\setlength{\droptitle}{-2em}

  \title{Introduction to R for Natural Resource Scientists}
    \pretitle{\vspace{\droptitle}\centering\huge}
  \posttitle{\par}
    \author{Ben Staton}
    \preauthor{\centering\large\emph}
  \postauthor{\par}
      \predate{\centering\large\emph}
  \postdate{\par}
    \date{with contributions from Henry Hershey}

\usepackage{booktabs}
\usepackage{amsthm}
\makeatletter
\def\thm@space@setup{%
  \thm@preskip=8pt plus 2pt minus 4pt
  \thm@postskip=\thm@preskip
}
\makeatother

\usepackage{amsthm}
\newtheorem{theorem}{Theorem}[chapter]
\newtheorem{lemma}{Lemma}[chapter]
\theoremstyle{definition}
\newtheorem{definition}{Definition}[chapter]
\newtheorem{corollary}{Corollary}[chapter]
\newtheorem{proposition}{Proposition}[chapter]
\theoremstyle{definition}
\newtheorem{example}{Example}[chapter]
\theoremstyle{definition}
\newtheorem{exercise}{Exercise}[chapter]
\theoremstyle{remark}
\newtheorem*{remark}{Remark}
\newtheorem*{solution}{Solution}
\begin{document}
\maketitle

{
\setcounter{tocdepth}{1}
\tableofcontents
}
\chapter*{Overview}\label{overview}
\addcontentsline{toc}{chapter}{Overview}

This is intended to be a first course in R programming. It is by no
means comprehensive, but instead attempts to introduce the main topics
needed to get a beginner up and running with applying R to their own
work. There will be no prior knowledge assumed on the part of the
student regarding R or programming. In the later chapters, e.g.,
Chapters \ref{ch3} and \ref{ch4}, an understanding of statistics at the
introductory undergraduate level would be helpful.

\chapter{Introduction to the R Environment}\label{ch1}

\begin{center}\rule{0.5\linewidth}{\linethickness}\end{center}

\section*{Chapter Overview}\label{chapter-overview}
\addcontentsline{toc}{section}{Chapter Overview}

In this first chapter, you will get familiar with the basics of using R.
You will learn:

\begin{itemize}
\tightlist
\item
  the use of R as a basic calculator
\item
  some basic object types
\item
  some basic data classes
\item
  some basic data structures
\item
  how to read in data
\item
  how to write out data
\item
  how to write your own functions
\end{itemize}

\section{Getting Started: Install R and
RStudio}\label{getting-started-install-r-and-rstudio}

First off, you will need to get R and RStudio\footnote{While it is
  possible to run R on it's own, it rather clunky and you are strongly
  advised to use RStudio given its compactness, neat features, code
  tools (like syntax and parentheses highlighting). This workshop will
  assume you are using RStudio} onto your computer. Please see
\textbf{Appendix A} for details on installing these programs on your
operating system.

\section{The R Studio Interface}\label{the-r-studio-interface}

Once you open up R Studio for the first time, you will see three panes:
the left hand side is the \textbf{console} where results from executed
commands are printed, and the two panes on the right are for additional
information to help you code more efficiently - don't worry too much
about what these are at the moment. For now, focus your attention on the
console.

\subsection{Write Some Simple Code}\label{write-some-simple-code}

To start off, you will just use R as a calculator. Type these commands
(not the lines with \texttt{\#\#}, those are output\footnote{The
  formatting used here includes \texttt{\#\#} on output to denote code
  and output separately. You won't see the \texttt{\#\#} show up in your
  console.}) one at a time and hit \textbf{CTRL + ENTER} to run it. The
spaces don't matter at all, they are used here for clairity and for
styling.\footnote{To learn more about standard R code styling, check out
  Hadley Wickham's \href{http://adv-r.had.co.nz/Style.html}{great
  chapter} about it.}

\begin{Shaded}
\begin{Highlighting}[]
\DecValTok{3} \OperatorTok{+}\StringTok{ }\DecValTok{3}
\end{Highlighting}
\end{Shaded}

\begin{verbatim}
## [1] 6
\end{verbatim}

\begin{Shaded}
\begin{Highlighting}[]
\DecValTok{12}\OperatorTok{/}\DecValTok{4}
\end{Highlighting}
\end{Shaded}

\begin{verbatim}
## [1] 3
\end{verbatim}

Notice that when you run each line, it prints the command and the output
to the console.

R is an \textbf{object oriented language}, which means that you fill
objects with data do things with them. Make an object called \texttt{x}
that stores the result of the calculation \texttt{3\ +\ 3} (type this
and run using \textbf{CTRL + ENTER}):

\begin{Shaded}
\begin{Highlighting}[]
\NormalTok{x =}\StringTok{ }\DecValTok{3} \OperatorTok{+}\StringTok{ }\DecValTok{3}
\end{Highlighting}
\end{Shaded}

Notice that running this line did not return a value as before. This is
because in that line you are \textbf{assigning} a value to the object
\texttt{x}. You can view the contents of \texttt{x} by typing its name
alone and running just that:

\begin{Shaded}
\begin{Highlighting}[]
\NormalTok{x}
\end{Highlighting}
\end{Shaded}

\begin{verbatim}
## [1] 6
\end{verbatim}

When used this way, the \texttt{=} sign denotes assignment of the value
on the right-hand side to an object with the name on the left-hand side.
The \texttt{\textless{}-} serves this same purpose so in this context
the two are interchangable:

\begin{Shaded}
\begin{Highlighting}[]
\NormalTok{y <-}\StringTok{ }\DecValTok{2} \OperatorTok{+}\StringTok{ }\DecValTok{5}
\end{Highlighting}
\end{Shaded}

You can highlight smaller sections of a line to run as well. For example
after creating \texttt{y} above, press the \textbf{up arrow} to see the
line you just ran, highlight just the \texttt{y}, and press \textbf{CTRL
+ ENTER}. From this point forward, the verb ``run'' means execute some
code using \textbf{CTRL + ENTER}.

You can use your objects together to make a new object:

\begin{Shaded}
\begin{Highlighting}[]
\NormalTok{z =}\StringTok{ }\NormalTok{y }\OperatorTok{-}\StringTok{ }\NormalTok{x}
\end{Highlighting}
\end{Shaded}

\section{Saving Your Code: Scripts}\label{saving-your-code-scripts}

If you closed R at this moment, your work would be lost. Running code in
the console like you have just done \textbf{does not save a record of
your work}. To save R code, you must use what is called a
\textbf{script}, which is a plain-text file with the extension
\texttt{.R}. To create a new script file, go to \emph{File
\textgreater{} New File \textgreater{} R Script}, or use the keyboard
shortcut \textbf{CTRL + SHIFT + N}. A new pane will open called the
\textbf{source} pane - this is where you will edit your code and save
your progress. R Scripts are a key feature of reproducible research with
R, given that if they are well-written they can present a complete
roadmap of your statistical analysis and workflow.

\section{The Working Directory}\label{the-working-directory}

You will want to save your hard work. A key part of doing saving your
work is thinking about \textbf{where} you save it. In R, a key concept
is the \textbf{working directory}. This is the location (i.e., folder)
on your computer where your current R session will ``talk to'' by
default. The working directory is where R will read files from and write
files to by default, and is where all of your data should be stored for
your analysis in R. Because you'll likely be visiting it often, it
should probably be somewhere that is easy to remember and not too deeply
buried in your computer's file system.

Save your script somewhere like
\texttt{C:/Users/YOU/Documents/R-Workshop/Chapter1} on your computer. To
set the working directory to this location, you have three options:

\begin{enumerate}
\def\labelenumi{\arabic{enumi}.}
\item
  \textbf{Go to Session \textgreater{} Set Working Directory
  \textgreater{} Source File Location}. This will set the working
  directory to the location of the file that is currently open in your
  source pane.
\item
  \textbf{Go to Session \textgreater{} Set Working Directory
  \textgreater{} Choose Directory}. This will open an interactive file
  selection window to allow you to navigate to the desired directory.
\item
  \textbf{Use code}. In the console, you can type
  \texttt{setwd("C:/Users/YOU/Documents/R-Workshop/Chapter1")}. If at
  any point you want to know where your current working directory is set
  to, you can either look at the top of the console pane, which shows
  the full path or by running \texttt{getwd()} in the console.
\end{enumerate}

\textbf{The main benefits of using a working directory are}:

\begin{itemize}
\tightlist
\item
  Files are read from and written to a consistent and predictable place
  everytime
\item
  Everything for your analysis is organized into one place
\item
  You don't have to continously type file paths to your work. If
  \texttt{file.txt} is a file in your current working directory, you can
  reference it your R session using \texttt{"file.txt"} rather than with
  \texttt{"C:/Users/YOU/Documents/R-Workshop/Chapter1/file.txt"} each
  time.
\end{itemize}

\section{R Object Types}\label{r-object-types}

R has a variety of object types that you will need to become familiar
with.

\subsection{Functions}\label{functions}

Much of your work in R will involve functions. A function is called
using the syntax:

\begin{Shaded}
\begin{Highlighting}[]
\KeywordTok{fun}\NormalTok{(}\DataTypeTok{arg1 =}\NormalTok{ value1, }\DataTypeTok{arg2 =}\NormalTok{ value2)}
\end{Highlighting}
\end{Shaded}

Here, \texttt{fun} is the \textbf{function name} and \texttt{arg1} and
\texttt{arg2} are called \textbf{arguments}. Functions take input in the
form of the arguments, do some task with them, then return some output.
The parentheses are a sure sign that \texttt{fun} is a function.

We have passed the function two arguments by name: all functions have
arguments, all arguments have names, and there is always a default order
to the arguments. If you memorize the argument order of functions you
use frequently, you don't have to specify the argument name:

\begin{Shaded}
\begin{Highlighting}[]
\KeywordTok{fun}\NormalTok{(value1, value2)}
\end{Highlighting}
\end{Shaded}

would give the same result as the command above in which the argument
names were specified.

Here's a real example:

\begin{Shaded}
\begin{Highlighting}[]
\KeywordTok{print}\NormalTok{(}\DataTypeTok{x =}\NormalTok{ z)}
\end{Highlighting}
\end{Shaded}

\begin{verbatim}
## [1] 1
\end{verbatim}

The function is \texttt{print}, the argument is \texttt{x}, and the
value we have supplied the argument is the object \texttt{z}. The task
that \texttt{print} does is to print the value of \texttt{z} to the
console.

R has lots of built-in information to help you learn how to use a
function. Take a look at the help file for the mean function. Run
\texttt{?mean} in the console: a window on the right-hand side of the R
Studio interface should open. The help file tells you what goes into a
function and what comes out. For more complex functions it also tells
you what all of the options (i.e., arguments) can do. Help files can be
a bit intimidating to interpret at first, but they are all organized the
same and once you learn their layout you will know where to go to find
the information you're looking for.

\subsection{Vectors}\label{vectors}

Vectors are one of the most common data structures. A vector is a set of
numbers going in only one dimension. Each position in a vector is termed
an \textbf{element}, and the number of elements is termed the
\textbf{length} of the vector. Here are some ways to make some vectors
with different elements, all of length five:

\begin{Shaded}
\begin{Highlighting}[]
\CommentTok{# this is a comment. R will ignore all text on a line after a #}
\CommentTok{# the ; means run everything after it on a new line}

\CommentTok{# count up by 1}
\NormalTok{month =}\StringTok{ }\DecValTok{2}\OperatorTok{:}\DecValTok{6}\NormalTok{; month}
\end{Highlighting}
\end{Shaded}

\begin{verbatim}
## [1] 2 3 4 5 6
\end{verbatim}

\begin{Shaded}
\begin{Highlighting}[]
\CommentTok{# count up by 2}
\NormalTok{day =}\StringTok{ }\KeywordTok{seq}\NormalTok{(}\DataTypeTok{from =} \DecValTok{1}\NormalTok{, }\DataTypeTok{to =} \DecValTok{9}\NormalTok{, }\DataTypeTok{by =} \DecValTok{2}\NormalTok{); day}
\end{Highlighting}
\end{Shaded}

\begin{verbatim}
## [1] 1 3 5 7 9
\end{verbatim}

\begin{Shaded}
\begin{Highlighting}[]
\CommentTok{# repeat the same number (repeat 2018 5 times)}
\NormalTok{year =}\StringTok{ }\KeywordTok{rep}\NormalTok{(}\DecValTok{2018}\NormalTok{, }\DecValTok{5}\NormalTok{); year}
\end{Highlighting}
\end{Shaded}

\begin{verbatim}
## [1] 2018 2018 2018 2018 2018
\end{verbatim}

The \texttt{{[}1{]}} that shows up is a element position, more on this
later (see Section \ref{sub}). If you wish to know how many elements are
in a vector, use \texttt{length}:

\begin{Shaded}
\begin{Highlighting}[]
\KeywordTok{length}\NormalTok{(year)}
\end{Highlighting}
\end{Shaded}

\begin{verbatim}
## [1] 5
\end{verbatim}

You can also create a vector ``by-hand'' using the \texttt{c}
function\footnote{The \texttt{c} stands for \textbf{concatenate}, which
  basically means combine many smaller objects into one larger object}:

\begin{Shaded}
\begin{Highlighting}[]
\CommentTok{# a numeric vector}
\NormalTok{number =}\StringTok{ }\KeywordTok{c}\NormalTok{(}\DecValTok{4}\NormalTok{, }\DecValTok{7}\NormalTok{, }\DecValTok{8}\NormalTok{, }\DecValTok{10}\NormalTok{, }\DecValTok{15}\NormalTok{); number}
\end{Highlighting}
\end{Shaded}

\begin{verbatim}
## [1]  4  7  8 10 15
\end{verbatim}

\begin{Shaded}
\begin{Highlighting}[]
\CommentTok{# a character vector}
\NormalTok{pond =}\StringTok{ }\KeywordTok{c}\NormalTok{(}\StringTok{"F11"}\NormalTok{, }\StringTok{"S28"}\NormalTok{, }\StringTok{"S30"}\NormalTok{, }\StringTok{"S8"}\NormalTok{, }\StringTok{'S11'}\NormalTok{); pond}
\end{Highlighting}
\end{Shaded}

\begin{verbatim}
## [1] "F11" "S28" "S30" "S8"  "S11"
\end{verbatim}

Note the difference between the numeric and character vectors. The the
terms ``numeric'' and ``character''" represent \textbf{data classes},
which specify the type of data the vector is holding:

\begin{itemize}
\tightlist
\item
  A \textbf{numeric vector} stores numbers. You can do math with numeric
  vectors
\item
  A \textbf{character vector} stores what are essentially letters. You
  can't do math with letters. A character vector is easy to spot because
  the elements will be wrapped with quotes\footnote{\texttt{"\ "} or
    \texttt{\textquotesingle{}\ \textquotesingle{}} both work as long as
    you use the same on the front and end of the element}.
\end{itemize}

A vector can only hold one data class at a time:

\begin{Shaded}
\begin{Highlighting}[]
\NormalTok{v =}\StringTok{ }\KeywordTok{c}\NormalTok{(}\DecValTok{1}\NormalTok{,}\DecValTok{2}\NormalTok{,}\DecValTok{3}\NormalTok{,}\StringTok{"a"}\NormalTok{); v}
\end{Highlighting}
\end{Shaded}

\begin{verbatim}
## [1] "1" "2" "3" "a"
\end{verbatim}

Notice how all the elements now have quotes around them. The numbers
have been \textbf{coerced} to characters\footnote{The coersion works
  this way because numbers can be expressed as characters, but a letter
  cannot be unambiguously be expressed as a number.}. If we attempted to
calculate the sum of our vector:

\begin{Shaded}
\begin{Highlighting}[]
\KeywordTok{sum}\NormalTok{(v)}
\end{Highlighting}
\end{Shaded}

\begin{verbatim}
## Error in sum(v): invalid 'type' (character) of argument
\end{verbatim}

we would find that it is impossible in its current form.

\subsection{Matrices}\label{matrices}

Matrices act just like vectors, but they are in two dimensions, i.e.,
they have both rows and columns. One easy way to make a matrix is by
combining vectors you have already made:

\begin{Shaded}
\begin{Highlighting}[]
\CommentTok{# combine vectors by column (each vector will become a column)}
\NormalTok{m1 =}\StringTok{ }\KeywordTok{cbind}\NormalTok{(month, day, year, number); m1}
\end{Highlighting}
\end{Shaded}

\begin{verbatim}
##      month day year number
## [1,]     2   1 2018      4
## [2,]     3   3 2018      7
## [3,]     4   5 2018      8
## [4,]     5   7 2018     10
## [5,]     6   9 2018     15
\end{verbatim}

\begin{Shaded}
\begin{Highlighting}[]
\CommentTok{# combine vectors by row (each vector will become a row)}
\NormalTok{m2 =}\StringTok{ }\KeywordTok{rbind}\NormalTok{(month, day, year, number); m2}
\end{Highlighting}
\end{Shaded}

\begin{verbatim}
##        [,1] [,2] [,3] [,4] [,5]
## month     2    3    4    5    6
## day       1    3    5    7    9
## year   2018 2018 2018 2018 2018
## number    4    7    8   10   15
\end{verbatim}

Just like vectors, matrices can hold only one data class (note the
coersion of numbers to characters):

\begin{Shaded}
\begin{Highlighting}[]
\KeywordTok{cbind}\NormalTok{(m1, pond)}
\end{Highlighting}
\end{Shaded}

\begin{verbatim}
##      month day year   number pond 
## [1,] "2"   "1" "2018" "4"    "F11"
## [2,] "3"   "3" "2018" "7"    "S28"
## [3,] "4"   "5" "2018" "8"    "S30"
## [4,] "5"   "7" "2018" "10"   "S8" 
## [5,] "6"   "9" "2018" "15"   "S11"
\end{verbatim}

\subsection{Data Frames}\label{data-frames}

Many data sets you will work with require storing different data classes
in different columns, which would rule out the use of a matrix. This is
where \textbf{data frames} come in:

\begin{Shaded}
\begin{Highlighting}[]
\NormalTok{df1 =}\StringTok{ }\KeywordTok{data.frame}\NormalTok{(month, day, year, number, pond); df1}
\end{Highlighting}
\end{Shaded}

\begin{verbatim}
##   month day year number pond
## 1     2   1 2018      4  F11
## 2     3   3 2018      7  S28
## 3     4   5 2018      8  S30
## 4     5   7 2018     10   S8
## 5     6   9 2018     15  S11
\end{verbatim}

Notice the lack of quotation marks which indicates that all variables
(i.e., columns) are stored as their original data class.

It is important to know what kind of object type you are using, since R
treats them differently. For example, some functions can only use a
certain object type. The same holds true for data classes (numeric
vs.~character).You can quickly determine what kind of object you are
dealing with by using the \texttt{class} function. Simply run
\texttt{class(object.name)}:

\begin{Shaded}
\begin{Highlighting}[]
\KeywordTok{class}\NormalTok{(day); }\KeywordTok{class}\NormalTok{(pond); }\KeywordTok{class}\NormalTok{(m1); }\KeywordTok{class}\NormalTok{(df1)}
\end{Highlighting}
\end{Shaded}

\begin{verbatim}
## [1] "numeric"
\end{verbatim}

\begin{verbatim}
## [1] "character"
\end{verbatim}

\begin{verbatim}
## [1] "matrix"
\end{verbatim}

\begin{verbatim}
## [1] "data.frame"
\end{verbatim}

\section{Factors}\label{factors}

At this point, it is worthwhile to introduce an additional data class:
factors. Notice the class of the \texttt{pond} variable in \texttt{df1}:

\begin{Shaded}
\begin{Highlighting}[]
\KeywordTok{class}\NormalTok{(df1}\OperatorTok{$}\NormalTok{pond)}
\end{Highlighting}
\end{Shaded}

\begin{verbatim}
## [1] "factor"
\end{verbatim}

The character vector \texttt{pond} was coerced to a factor when you
placed it in the data frame. A vector with a \textbf{factor} class is
like a character vector in that you see letters and that you can't do
math on it. However, a factor has additional properties: in particular,
it is a grouping variable. See what happens when you print the
\texttt{pond} variable:

\begin{Shaded}
\begin{Highlighting}[]
\NormalTok{df1}\OperatorTok{$}\NormalTok{pond}
\end{Highlighting}
\end{Shaded}

\begin{verbatim}
## [1] F11 S28 S30 S8  S11
## Levels: F11 S11 S28 S30 S8
\end{verbatim}

That looks weird, huh? A factor has levels, with each level being a
subcategory of the factor. You can see the unique levels of your factor
by running:

\begin{Shaded}
\begin{Highlighting}[]
\KeywordTok{levels}\NormalTok{(df1}\OperatorTok{$}\NormalTok{pond)}
\end{Highlighting}
\end{Shaded}

\begin{verbatim}
## [1] "F11" "S11" "S28" "S30" "S8"
\end{verbatim}

Additionally, factor levels have an assigned order (even if the levels
are totally nominal), which will become important in Chapter \ref{ch3}
when you learn how to fit linear models to groups of data, in which one
level is the ``reference'' group that all other groups are compared to.

If you're running into errors about R expecting character vectors, it
may be because they are actually stored as factors. When you make a data
frame, you'll often have the option to turn off the automatic factor
coersion. For example:

\begin{Shaded}
\begin{Highlighting}[]
\KeywordTok{data.frame}\NormalTok{(month, day, year, number, pond, }\DataTypeTok{stringsAsFactors =}\NormalTok{ F)}
\KeywordTok{read.csv}\NormalTok{(}\StringTok{"streams.csv"}\NormalTok{, }\DataTypeTok{stringsAsFactors =}\NormalTok{ F)  }\CommentTok{# see below for details on read.csv}
\end{Highlighting}
\end{Shaded}

will result in character vectors remaining that way as opposed to being
coerced to factors. This can be preferable if you are doing many string
manipulations, as character vectors are often easier to work with than
factors.

\section{Vector Math}\label{vector-math}

R does vectorized calculations. This means that if supplied with two
numeric vectors of equal length and a mathematical operator, R will
perform the calculation on each pair of elements. For example, if you
wanted to add the two vectors vector \texttt{day} and \texttt{month},
then you would just run:

\begin{Shaded}
\begin{Highlighting}[]
\NormalTok{dm =}\StringTok{ }\NormalTok{day }\OperatorTok{+}\StringTok{ }\NormalTok{month; dm}
\end{Highlighting}
\end{Shaded}

\begin{verbatim}
## [1]  3  6  9 12 15
\end{verbatim}

You typically should ensure that the vectors you are doing math with are
of equal lengths.

You could do the same calculation to each element (e.g., divide each
element by 2) with:

\begin{Shaded}
\begin{Highlighting}[]
\NormalTok{dm}\OperatorTok{/}\DecValTok{2}
\end{Highlighting}
\end{Shaded}

\begin{verbatim}
## [1] 1.5 3.0 4.5 6.0 7.5
\end{verbatim}

\section{Data Queries/Subsets/Retrievals
\{sub\}}\label{data-queriessubsetsretrievals-sub}

This perhaps the most important and versatile skills to know in R. So
you have an object with data in it and you want to use it for analysis.
But you don't want the whole dataset. You want to use just a few rows or
just a few columns, or perhaps you need just a single element from a
vector. This section is devoted to ways you can extract certain parts of
a data object (the terms \textbf{query} and \textbf{subset} are often
used interchangeably to describe this task). There are three main
methods:

\begin{enumerate}
\def\labelenumi{\arabic{enumi}.}
\tightlist
\item
  \textbf{By Index} -- This method allows you to pull out specific
  rows/columns by their location in an object. However, you must know
  exactly where in the object the desired data are. An \textbf{index} is
  a location of a element in a data object, like the element position or
  the position of a specific row or column. To subset by index, you
  specify the object, then what rows, then what columns. The syntax for
  subsetting a vector by index is \texttt{vector{[}element{]}} and for a
  matrix it is \texttt{matrix{[}row,column{]}}. Here are some examples
\end{enumerate}

\begin{Shaded}
\begin{Highlighting}[]
\CommentTok{# show all of day, then subset the third element}
\NormalTok{day; day[}\DecValTok{3}\NormalTok{]}
\end{Highlighting}
\end{Shaded}

\begin{verbatim}
## [1] 1 3 5 7 9
\end{verbatim}

\begin{verbatim}
## [1] 5
\end{verbatim}

\begin{Shaded}
\begin{Highlighting}[]
\CommentTok{# show all of m1, then subset the cell in row 1 col 3 }
\NormalTok{m1; m1[}\DecValTok{1}\NormalTok{,}\DecValTok{3}\NormalTok{]}
\end{Highlighting}
\end{Shaded}

\begin{verbatim}
##      month day year number
## [1,]     2   1 2018      4
## [2,]     3   3 2018      7
## [3,]     4   5 2018      8
## [4,]     5   7 2018     10
## [5,]     6   9 2018     15
\end{verbatim}

\begin{verbatim}
## year 
## 2018
\end{verbatim}

\begin{Shaded}
\begin{Highlighting}[]
\CommentTok{# show all of df1, then subset the entire first column}
\NormalTok{df1; df1[,}\DecValTok{1}\NormalTok{]}
\end{Highlighting}
\end{Shaded}

\begin{verbatim}
##   month day year number pond
## 1     2   1 2018      4  F11
## 2     3   3 2018      7  S28
## 3     4   5 2018      8  S30
## 4     5   7 2018     10   S8
## 5     6   9 2018     15  S11
\end{verbatim}

\begin{verbatim}
## [1] 2 3 4 5 6
\end{verbatim}

Note this last line: the \texttt{{[},1{]}} says ``keep all the rows, but
take only the first column''.

Here is another example:

\begin{Shaded}
\begin{Highlighting}[]
\CommentTok{# show m1, then subset the 1st, 2nd, and 4th rows and every column}
\NormalTok{m1; m1[}\KeywordTok{c}\NormalTok{(}\DecValTok{1}\NormalTok{,}\DecValTok{2}\NormalTok{,}\DecValTok{4}\NormalTok{),]}
\end{Highlighting}
\end{Shaded}

\begin{verbatim}
##      month day year number
## [1,]     2   1 2018      4
## [2,]     3   3 2018      7
## [3,]     4   5 2018      8
## [4,]     5   7 2018     10
## [5,]     6   9 2018     15
\end{verbatim}

\begin{verbatim}
##      month day year number
## [1,]     2   1 2018      4
## [2,]     3   3 2018      7
## [3,]     5   7 2018     10
\end{verbatim}

Notice how you can pass a vector of row indices here to exclude the
3\textsuperscript{rd} and 5\textsuperscript{th} rows.

\textbf{2. By name} -- This method allows you to pull out a specific
column of data based on what the column name is. Of course, the column
must have a name first. The name method uses the \texttt{\$} operator:

\begin{Shaded}
\begin{Highlighting}[]
\NormalTok{df1}\OperatorTok{$}\NormalTok{month}
\end{Highlighting}
\end{Shaded}

\begin{verbatim}
## [1] 2 3 4 5 6
\end{verbatim}

You can combine these two methods:

\begin{Shaded}
\begin{Highlighting}[]
\NormalTok{df1}\OperatorTok{$}\NormalTok{month[}\DecValTok{3}\NormalTok{]}
\end{Highlighting}
\end{Shaded}

\begin{verbatim}
## [1] 4
\end{verbatim}

\begin{Shaded}
\begin{Highlighting}[]
\CommentTok{# or}
\NormalTok{df1[,}\StringTok{"year"}\NormalTok{]}
\end{Highlighting}
\end{Shaded}

\begin{verbatim}
## [1] 2018 2018 2018 2018 2018
\end{verbatim}

The by name (\texttt{\$}) method is useful because it can be used to add
columns to a data frame:

\begin{Shaded}
\begin{Highlighting}[]
\NormalTok{df1}\OperatorTok{$}\NormalTok{dm =}\StringTok{ }\NormalTok{df1}\OperatorTok{$}\NormalTok{day }\OperatorTok{+}\StringTok{ }\NormalTok{df1}\OperatorTok{$}\NormalTok{month; df1}
\end{Highlighting}
\end{Shaded}

\begin{verbatim}
##   month day year number pond dm
## 1     2   1 2018      4  F11  3
## 2     3   3 2018      7  S28  6
## 3     4   5 2018      8  S30  9
## 4     5   7 2018     10   S8 12
## 5     6   9 2018     15  S11 15
\end{verbatim}

\begin{enumerate}
\def\labelenumi{\arabic{enumi}.}
\setcounter{enumi}{2}
\tightlist
\item
  \textbf{Logical Subsetting} -- This is perhaps the most flexible
  method, and is described in Section \ref{logsub}.
\end{enumerate}

\section*{EXERCISE 1A}\label{exercise-1a}
\addcontentsline{toc}{section}{EXERCISE 1A}

Take a break to apply what you've learned so far. In this exercise,
you'll be using what you learned about creating objects and the
differences between different data classes. You will be entering this
data frame into R by hand and doing some basic data subsets.

\begin{longtable}[]{@{}cccc@{}}
\toprule
\begin{minipage}[b]{0.11\columnwidth}\centering\strut
Lake\strut
\end{minipage} & \begin{minipage}[b]{0.09\columnwidth}\centering\strut
Area\strut
\end{minipage} & \begin{minipage}[b]{0.09\columnwidth}\centering\strut
Time\strut
\end{minipage} & \begin{minipage}[b]{0.09\columnwidth}\centering\strut
Fish\strut
\end{minipage}\tabularnewline
\midrule
\endhead
\begin{minipage}[t]{0.11\columnwidth}\centering\strut
Big\strut
\end{minipage} & \begin{minipage}[t]{0.09\columnwidth}\centering\strut
100\strut
\end{minipage} & \begin{minipage}[t]{0.09\columnwidth}\centering\strut
1000\strut
\end{minipage} & \begin{minipage}[t]{0.09\columnwidth}\centering\strut
643\strut
\end{minipage}\tabularnewline
\begin{minipage}[t]{0.11\columnwidth}\centering\strut
Small\strut
\end{minipage} & \begin{minipage}[t]{0.09\columnwidth}\centering\strut
25\strut
\end{minipage} & \begin{minipage}[t]{0.09\columnwidth}\centering\strut
1200\strut
\end{minipage} & \begin{minipage}[t]{0.09\columnwidth}\centering\strut
203\strut
\end{minipage}\tabularnewline
\begin{minipage}[t]{0.11\columnwidth}\centering\strut
Square\strut
\end{minipage} & \begin{minipage}[t]{0.09\columnwidth}\centering\strut
45\strut
\end{minipage} & \begin{minipage}[t]{0.09\columnwidth}\centering\strut
1400\strut
\end{minipage} & \begin{minipage}[t]{0.09\columnwidth}\centering\strut
109\strut
\end{minipage}\tabularnewline
\begin{minipage}[t]{0.11\columnwidth}\centering\strut
Circle\strut
\end{minipage} & \begin{minipage}[t]{0.09\columnwidth}\centering\strut
30\strut
\end{minipage} & \begin{minipage}[t]{0.09\columnwidth}\centering\strut
1600\strut
\end{minipage} & \begin{minipage}[t]{0.09\columnwidth}\centering\strut
15\strut
\end{minipage}\tabularnewline
\bottomrule
\end{longtable}

\begin{enumerate}
\def\labelenumi{\arabic{enumi}.}
\item
  Create a new file called \texttt{Ex\_1A.R}. Set the working directory
  to the folder containing \texttt{Ex\_1A.R}.
\item
  Enter these data into vectors. Call the vectors whatever you would
  like. Should you enter the data as vectors by rows, or by columns?
  (\emph{Hint: remember the properties of vectors}).
\item
  Combine your vectors into a data frame. Why should you use a data
  frame instead of a matrix?
\item
  Subset all of the data from Small Lake.
\item
  Subset the area for all of the lakes.
\item
  Subset the number of fish for Big and Square lakes.
\item
  You realize that you sampled 209 fish at Square Lake, not 109. Fix the
  mistake. There are two ways to do this, can you think of them both?
  Which do you think is better?
\item
  Save your script. Close R and re-open your script to see that it was
  saved.
\end{enumerate}

\section{Read External Data Files}\label{read}

It is rare that you will enter data by hand as you did in Exercise 1A.
Often, you have a dataset that you wish to analyze or manipulate. R has
several ways to read information from data files and in this workshop,
we will be using a common and simple method: reading in \texttt{.csv}
files. \texttt{.csv} files are data files that separate columns with
commas\footnote{Note that if your computer is configured for a
  Spanish-speaking country, Microsoft Excel might convert decimals to
  commas. This can really mess with reading in data - I would suggest
  changing the language of Excel if you find this to be the case.}. If
your data are in a Microsoft Excel spreadsheet, you can save your
spreadsheet file as a CSV file (\emph{File \textgreater{} Save As
\textgreater{} Save as Type\textgreater{} CSV (Comma Delimited)}).
Several dialog boxes will open asking if you are sure you want to save
it as a \texttt{.csv} file.

The syntax for reading in a \texttt{.csv} file is:

\begin{Shaded}
\begin{Highlighting}[]
\NormalTok{dat =}\StringTok{ }\KeywordTok{read.csv}\NormalTok{(}\StringTok{"FileName.csv"}\NormalTok{)}
\end{Highlighting}
\end{Shaded}

The data files for this workshop are found in the GitHub repository:
\url{https://github.com/bstaton1/au-r-workshop-data/tree/master}.
Navigate to this repository, find the file called \texttt{streams.csv},
download that file\footnote{To download a single file from GitHub, click
  on the file, then click ``Raw'' in the toolbar on the topright, then
  right-click anywhere in the document and click ``Save As\ldots{}''.
  You can also download a zip folder with all of the \texttt{.csv}
  files, go to ``Clone or Download''}, and place it in your working
directory for this session. This document assumes your working directory
is \texttt{C:/Users/YOU/Documents/R-Workshop/Chapter1}, though the one
you are actually using may be slightly different. We can read the
contents of \texttt{streams.csv} into R using this code (ensure your
working directory is set first):

\begin{Shaded}
\begin{Highlighting}[]
\NormalTok{dat =}\StringTok{ }\KeywordTok{read.csv}\NormalTok{(}\StringTok{"streams.csv"}\NormalTok{)}
\end{Highlighting}
\end{Shaded}

\textbf{If the data set is in your working directory, all you need to
provide is the file name.} If you do not get an error, congratulations!
However, if you get an error that looks like this:

\begin{verbatim}
## Warning in file(file, "rt"): cannot open file 'streams.csv': No such file
## or directory
\end{verbatim}

\begin{verbatim}
## Error in file(file, "rt"): cannot open the connection
\end{verbatim}

then fear not. This has must be among the most common errors encoutered
by R users world-wide. It simply means the file you told R to look for
doesn't exist where you told R to find it. Here is a trouble-shooting
guide to this error:

\begin{enumerate}
\def\labelenumi{\arabic{enumi}.}
\item
  The exact case and spelling matters, as well as do the quotes and
  \texttt{.csv} at the end. Ensure the file name is typed correctly.
\item
  Check what files are in your working directory: run \texttt{dir()}.
  This will return a vector with the names of the files located in your
  working directory. Is the file you told R was there truly in there? Is
  your working directory set to where you thought it was?
\item
  If the file is not in your working directory, and you wish to keep it
  that way, you must point R to where that file is. Pretend for a moment
  that we put the data file in \texttt{/R-Workshop/DataHere}. Here are
  two examples that would both read in the file in this case:
\end{enumerate}

\begin{Shaded}
\begin{Highlighting}[]
\NormalTok{dat =}\StringTok{ }\KeywordTok{read.csv}\NormalTok{(}\StringTok{"C:/Users/YOU/Documents/R-Workshop/Data/streams.csv"}\NormalTok{)}
\NormalTok{dat =}\StringTok{ }\KeywordTok{read.csv}\NormalTok{(}\StringTok{"../Data/streams"}\NormalTok{)}
\end{Highlighting}
\end{Shaded}

The first line is the full file path to the file in question. The second
line uses a \textbf{relative path}: the \texttt{../} says ``look one
folder up from the working directory for a folder called \texttt{Data}
then find a file called \texttt{streams.csv} in it.''

If you did not get any errors, then the data are in the object you named
(\texttt{dat} here) and that object is a data frame. Do not proceed
until you are able to \texttt{read.csv} to run successfully.

\textbf{A few things to note about reading in \texttt{.csv} files}:

\begin{itemize}
\item
  If even a single character is found in a numeric column in
  \texttt{FileName.csv}, the \emph{entire column} will be coerced to a
  character/factor data class after it is read in (i.e., no more math
  with data on that column until you remove the character). A common
  error is to have a \texttt{\#VALUE!} record left over from an invalid
  Excel function result. You must remove all of these occurences in
  order to use that column as numeric. Characters include anything other
  than a number ({[}0-9{]}) and a period when used as a decimal. None of
  these characters: \texttt{!@\#\$\%\^{}\&*()\_-+={[}a-z{]};{[}A-Z{]}}
  should never be found in a column you wish to do math with (e.g., take
  the mean of that column).
\item
  If a record (i.e., cell) is truly missing and you wish R to treat it
  that way (i.e., as an \texttt{NA}), you have three options:

  \begin{itemize}
  \tightlist
  \item
    Hard code an \texttt{NA} into that cell in Excel
  \item
    Leave that cell completely empty
  \item
    Enter in some other character (e.g., \texttt{"."}) alone in all
    cells that are meant to be coded as \texttt{NA} in R and use the
    \texttt{na.strings\ =\ "."} argument of \texttt{read.csv}.
  \end{itemize}
\item
  If a cell in the first row with text contains a space between two
  words, R will insert a \texttt{"."} between the words.
\item
  If your data have column names (i.e., characters) in the first row, R
  will bring those in as column names by default.
\item
  R brings in CSV files in as data frames by default.
\item
  If at some point you did ``Clear Contents'' in Microsoft Excel to
  delete rows or columns from your \texttt{.csv} file, these deleted
  rows/columns will be read in as all \texttt{NA}s, which can be
  annoying. To remove this problem, open the \texttt{.csv} file in
  Excel, then highlight and \textbf{delete} the rows/columns and save
  the file. Read it back into R again using \texttt{read.csv}.
\end{itemize}

\section{Explore the Data Set}\label{explore-the-data-set}

Have a look at the data. You could just run \texttt{dat} to view the
contents of the object, but it will show the whole thing, which may be
undesirable if the data set is large. To view the first handful of rows,
run \texttt{head(dat)} or the last handful of rows with
\texttt{tail(dat)}.

You will now use some basic functions to explore the streams data before
any analysis. The \texttt{summary} function is very useful for getting a
coarse look at how R has interpretted the data frame:

\begin{Shaded}
\begin{Highlighting}[]
\KeywordTok{summary}\NormalTok{(dat)}
\end{Highlighting}
\end{Shaded}

\begin{verbatim}
##        state    stream_width        flow       
##  Alabama  :5   Min.   :17.65   Min.   : 28.75  
##  Florida  :5   1st Qu.:46.09   1st Qu.: 65.50  
##  Georgia  :5   Median :61.80   Median : 95.64  
##  Tennessee:5   Mean   :60.88   Mean   : 91.49  
##                3rd Qu.:79.34   3rd Qu.:120.55  
##                Max.   :94.65   Max.   :149.54  
##                                NA's   :1
\end{verbatim}

You can see the spread of the numeric data and see the different levels
of the factor (\texttt{state}) as well as how many data points belong to
each category. Note that there is one \texttt{NA} in the variable called
\texttt{flow}.

To count the number of elements in a variable (or any vector), remember
the \texttt{length} function:

\begin{Shaded}
\begin{Highlighting}[]
\KeywordTok{length}\NormalTok{(dat}\OperatorTok{$}\NormalTok{stream_width)}
\end{Highlighting}
\end{Shaded}

\begin{verbatim}
## [1] 20
\end{verbatim}

Note that R counts missing values as elements as well:

\begin{Shaded}
\begin{Highlighting}[]
\KeywordTok{length}\NormalTok{(dat}\OperatorTok{$}\NormalTok{flow)}
\end{Highlighting}
\end{Shaded}

\begin{verbatim}
## [1] 20
\end{verbatim}

To get the dimensions of an object with more than one dimension (i.e., a
data frame or matrix) we can use the \texttt{dim} function. This returns
a vector with two elements: the first number is the number of rows and
the second is the number of columns. If you only want one of these, use
the \texttt{nrow} or \texttt{ncol} functions (but remember, only for
objects with more than one dimension; vectors don't have rows or
columns!).

\begin{Shaded}
\begin{Highlighting}[]
\KeywordTok{dim}\NormalTok{(dat); }\KeywordTok{nrow}\NormalTok{(dat); }\KeywordTok{ncol}\NormalTok{(dat)}
\end{Highlighting}
\end{Shaded}

\begin{verbatim}
## [1] 20  3
\end{verbatim}

\begin{verbatim}
## [1] 20
\end{verbatim}

\begin{verbatim}
## [1] 3
\end{verbatim}

You can extract the names of the variables (i.e., columns) in the data
frame using \texttt{colnames}:

\begin{Shaded}
\begin{Highlighting}[]
\KeywordTok{colnames}\NormalTok{(dat)}
\end{Highlighting}
\end{Shaded}

\begin{verbatim}
## [1] "state"        "stream_width" "flow"
\end{verbatim}

Calculate the mean of all of the flow records:

\begin{Shaded}
\begin{Highlighting}[]
\KeywordTok{mean}\NormalTok{(dat}\OperatorTok{$}\NormalTok{flow)}
\end{Highlighting}
\end{Shaded}

\begin{verbatim}
## [1] NA
\end{verbatim}

It returned an \texttt{NA} because there is an \texttt{NA} in the data
for this variable. There is a way to tell R to ignore this \texttt{NA}.
Include the argument \texttt{na.rm\ =\ TRUE} in the \texttt{mean}
function (separate arguments are always separated by commas). This is a
\textbf{logical} argument, meaning that it asks a question. It says ``do
you want to remove \texttt{NAs} before calculating the mean?''
\texttt{TRUE} means ``yes'' and \texttt{FALSE} means ``no.'' These can
be abbreviated as \texttt{T} or \texttt{F}. Many of R's functions have
the \texttt{na.rm} argument (e.g. \texttt{mean}, \texttt{sd},
\texttt{var}, \texttt{min}, \texttt{max}, \texttt{sum}, etc. - most
anything that collapses a vector into one number).

\begin{Shaded}
\begin{Highlighting}[]
\KeywordTok{mean}\NormalTok{(dat}\OperatorTok{$}\NormalTok{flow, }\DataTypeTok{na.rm =}\NormalTok{ T)}
\end{Highlighting}
\end{Shaded}

\begin{verbatim}
## [1] 91.49053
\end{verbatim}

which is the same as (i.e., the definition of the mean with the
\texttt{NA} removed):

\begin{Shaded}
\begin{Highlighting}[]
\KeywordTok{sum}\NormalTok{(dat}\OperatorTok{$}\NormalTok{flow, }\DataTypeTok{na.rm =}\NormalTok{ T)}\OperatorTok{/}\NormalTok{(}\KeywordTok{nrow}\NormalTok{(dat) }\OperatorTok{-}\StringTok{ }\DecValTok{1}\NormalTok{)}
\end{Highlighting}
\end{Shaded}

\begin{verbatim}
## [1] 91.49053
\end{verbatim}

What if you need to do something to more than one variable at a time?
One of the easiest ways to do this (though as with most things in R,
there are many) is by using the \texttt{apply} function. This function
applies the same summary function to individual subsets of a data object
at a time then returns the individual summaries all at once:

\begin{Shaded}
\begin{Highlighting}[]
\KeywordTok{apply}\NormalTok{(dat[,}\DecValTok{2}\OperatorTok{:}\DecValTok{3}\NormalTok{], }\DecValTok{2}\NormalTok{, }\DataTypeTok{FUN =}\NormalTok{ var, }\DataTypeTok{na.rm =}\NormalTok{ T)}
\end{Highlighting}
\end{Shaded}

\begin{verbatim}
## stream_width         flow 
##     581.1693    1337.3853
\end{verbatim}

The first argument is the data object you want to apply the function to.
The second argument (the number \texttt{2}) specifies that you want to
apply the function to columns, \texttt{1} would tell R to apply it to
rows. The \texttt{FUN} argument specifies what function you wish to
apply to each of the columns; here we are calculating the variance which
takes the \texttt{na.rm\ =\ T} argument. This use of \texttt{apply}
alone is very powerful and can help you get around having to write the
dreaded \texttt{for} loop (introduced in Chapter \ref{ch4}).

There is a whole family of \texttt{apply} functions, the base
\texttt{apply} is the most basic but a more sophisticated one is
\texttt{tapply}, which applies a function based on some grouping
variable (a factor). Calculate the mean stream width \textbf{separated
by state}:

\begin{Shaded}
\begin{Highlighting}[]
\KeywordTok{tapply}\NormalTok{(dat}\OperatorTok{$}\NormalTok{stream_width, dat}\OperatorTok{$}\NormalTok{state, mean)}
\end{Highlighting}
\end{Shaded}

\begin{verbatim}
##   Alabama   Florida   Georgia Tennessee 
##    53.664    63.588    54.996    71.290
\end{verbatim}

The first argument is the variable you want to apply the \texttt{mean}
function to, the second is the grouping variable, and the third is what
function you wish to apply. Try to commit this command to memory given
this is a pretty common task.

\section{Logical/Boolean Operators}\label{logicalboolean-operators}

To be an efficient and capable programmer in any language, you will need
to become familiar with how to implement numerical logic, i.e., the
boolean operators. These are very useful because they always return a
\texttt{TRUE} or a \texttt{FALSE}, off of which program-based decisions
can be made (e.g., whether to operate a given subroutine, whether to
keep certain rows, whether to print the output, etc.).

Define a simple object: \texttt{x\ =\ 5} . Note that this will write
over what was previously stored in the object \texttt{x}. We wish to ask
some questions of the new \texttt{x}, and the answer will be printed to
the console as a \texttt{TRUE} for ``yes'' and a \texttt{FALSE} for
``no''. Below are the common Boolean operators.

\subsubsection*{Equality}\label{equality}
\addcontentsline{toc}{subsubsection}{Equality}

To ask if \texttt{x} is exactly equal to 5, you run:

\begin{Shaded}
\begin{Highlighting}[]
\NormalTok{x }\OperatorTok{==}\StringTok{ }\DecValTok{5}
\end{Highlighting}
\end{Shaded}

\begin{verbatim}
## [1] TRUE
\end{verbatim}

Note the use of the double equals-sign to denote equality as opposed to
the single \texttt{=} as used in assignment when you ran
\texttt{x\ =\ 5} a minute ago.

\subsubsection*{Inequalities}\label{inequalities}
\addcontentsline{toc}{subsubsection}{Inequalities}

To ask if \texttt{x} is not equal to 5, you run:

\begin{Shaded}
\begin{Highlighting}[]
\NormalTok{x }\OperatorTok{!=}\StringTok{ }\DecValTok{5}
\end{Highlighting}
\end{Shaded}

\begin{verbatim}
## [1] FALSE
\end{verbatim}

To ask if \texttt{x} is less than 5, you run:

\begin{Shaded}
\begin{Highlighting}[]
\NormalTok{x }\OperatorTok{<}\StringTok{ }\DecValTok{5}
\end{Highlighting}
\end{Shaded}

\begin{verbatim}
## [1] FALSE
\end{verbatim}

To ask if \texttt{x} is less than \emph{or equal to} 5, you run:

\begin{Shaded}
\begin{Highlighting}[]
\NormalTok{x }\OperatorTok{<=}\StringTok{ }\DecValTok{5}
\end{Highlighting}
\end{Shaded}

\begin{verbatim}
## [1] TRUE
\end{verbatim}

Greater than works the same way, though with the \texttt{\textgreater{}}
symbol replaced.

\subsubsection*{And}\label{and}
\addcontentsline{toc}{subsubsection}{And}

Suppose you have two conditions, and you want to know if \textbf{both
are met}. For this you would use \textbf{and} by running:

\begin{Shaded}
\begin{Highlighting}[]
\NormalTok{x }\OperatorTok{>}\StringTok{ }\DecValTok{4} \OperatorTok{&}\StringTok{ }\NormalTok{x }\OperatorTok{<}\StringTok{ }\DecValTok{6}
\end{Highlighting}
\end{Shaded}

\begin{verbatim}
## [1] TRUE
\end{verbatim}

which asks if \texttt{x} is between 4 and 6.

\subsubsection*{Or}\label{or}
\addcontentsline{toc}{subsubsection}{Or}

Suppose you have two conditions, and you want to know if \textbf{either
are met}. For this you would use \textbf{or} by running:

\begin{Shaded}
\begin{Highlighting}[]
\NormalTok{x }\OperatorTok{<=}\StringTok{ }\DecValTok{5} \OperatorTok{|}\StringTok{ }\NormalTok{x }\OperatorTok{>}\StringTok{ }\DecValTok{5}
\end{Highlighting}
\end{Shaded}

\begin{verbatim}
## [1] TRUE
\end{verbatim}

which asks if \texttt{x} is less than or equal to 5 \textbf{or} greater
than 5 - you would be hard-pressed to find a real number that did not
meet these conditions!

\section{Logical Subsetting}\label{logsub}

A critical use of logical/Boolean operators is in the subsetting of data
objects. You can use a logical vector (i.e., one made of only
\texttt{TRUE} and \texttt{FALSE} elements) to tell R to extract only
those elements corresponding to the \texttt{TRUE} records. For example:

\begin{Shaded}
\begin{Highlighting}[]
\CommentTok{# here's logical vector: TRUE everywhere condition met}
\NormalTok{dat}\OperatorTok{$}\NormalTok{stream_width }\OperatorTok{>}\StringTok{ }\DecValTok{60}
\end{Highlighting}
\end{Shaded}

\begin{verbatim}
##  [1]  TRUE FALSE FALSE FALSE FALSE FALSE  TRUE FALSE  TRUE  TRUE FALSE
## [12] FALSE  TRUE  TRUE  TRUE FALSE FALSE  TRUE  TRUE  TRUE
\end{verbatim}

\begin{Shaded}
\begin{Highlighting}[]
\CommentTok{# insert it to see only the flows for the TRUE elmements}
\NormalTok{dat}\OperatorTok{$}\NormalTok{flow[dat}\OperatorTok{$}\NormalTok{stream_width }\OperatorTok{>}\StringTok{ }\DecValTok{60}\NormalTok{]}
\end{Highlighting}
\end{Shaded}

\begin{verbatim}
##  [1] 120.48 123.78  95.64  95.82 120.06 135.63 120.61 111.34 149.54 131.22
\end{verbatim}

gives all of the \texttt{flow} values for which \texttt{stream\_width}
is greater than 60.

To see all of the data from Alabama, you would run:

\begin{Shaded}
\begin{Highlighting}[]
\NormalTok{dat[dat}\OperatorTok{$}\NormalTok{state }\OperatorTok{==}\StringTok{ "Alabama"}\NormalTok{,]}
\end{Highlighting}
\end{Shaded}

\begin{verbatim}
##     state stream_width   flow
## 1 Alabama        81.68 120.48
## 2 Alabama        57.76  85.90
## 3 Alabama        48.32  73.38
## 4 Alabama        31.63  46.55
## 5 Alabama        48.93  80.91
\end{verbatim}

You will be frequently revisiting this skill throughout the workshop.

\section{if, else, and ifelse}\label{if-else-and-ifelse}

You can tell R to do something if the result of a question is
\texttt{TRUE}. This is a typical if-then statement:

\begin{Shaded}
\begin{Highlighting}[]
\ControlFlowTok{if}\NormalTok{ (x }\OperatorTok{==}\StringTok{ }\DecValTok{5}\NormalTok{) }\KeywordTok{print}\NormalTok{(}\StringTok{"x is equal to 5"}\NormalTok{)}
\end{Highlighting}
\end{Shaded}

\begin{verbatim}
## [1] "x is equal to 5"
\end{verbatim}

This says ``if \texttt{x} equals 5, then print the phrase `x is equal to
5' to the console''. If the logical returns a \texttt{FALSE}, then this
command does nothing:

\begin{Shaded}
\begin{Highlighting}[]
\ControlFlowTok{if}\NormalTok{ (x }\OperatorTok{!=}\StringTok{ }\DecValTok{5}\NormalTok{) }\KeywordTok{print}\NormalTok{(}\StringTok{"x is equal to 5"}\NormalTok{)}
\end{Highlighting}
\end{Shaded}

We can tell R to do multiple things if the logical is \texttt{TRUE} by
using curly braces:

\begin{Shaded}
\begin{Highlighting}[]
\ControlFlowTok{if}\NormalTok{ (x }\OperatorTok{==}\StringTok{ }\DecValTok{5}\NormalTok{) \{}
  \KeywordTok{print}\NormalTok{(}\StringTok{"x is equal to 5"}\NormalTok{)}
  \KeywordTok{print}\NormalTok{(}\StringTok{"you dummy, x is supposed to be 6"}\NormalTok{)}
\NormalTok{\}}
\end{Highlighting}
\end{Shaded}

\begin{verbatim}
## [1] "x is equal to 5"
## [1] "you dummy, x is supposed to be 6"
\end{verbatim}

You can always use curly braces to extend code across multiple lines
whereas it may have been intended to go on one line.

What if you want R to do something if the logical is \texttt{FALSE}?
Then you would use the \texttt{else} command:

\begin{Shaded}
\begin{Highlighting}[]
\ControlFlowTok{if}\NormalTok{ (x }\OperatorTok{>}\StringTok{ }\DecValTok{5}\NormalTok{) }\KeywordTok{print}\NormalTok{(}\StringTok{"x is greater than 5"}\NormalTok{) }\ControlFlowTok{else} \KeywordTok{print}\NormalTok{(}\StringTok{"x is not greater than 5"}\NormalTok{)}
\end{Highlighting}
\end{Shaded}

\begin{verbatim}
## [1] "x is not greater than 5"
\end{verbatim}

Or extend this same thing to multiple lines:

\begin{Shaded}
\begin{Highlighting}[]
\ControlFlowTok{if}\NormalTok{ (x }\OperatorTok{>}\StringTok{ }\DecValTok{5}\NormalTok{) \{}
  \KeywordTok{print}\NormalTok{(}\StringTok{"x is greater than 5"}\NormalTok{)}
\NormalTok{\} }\ControlFlowTok{else}\NormalTok{ \{}
  \KeywordTok{print}\NormalTok{(}\StringTok{"x is not greater than 5"}\NormalTok{)}
\NormalTok{\} }
\end{Highlighting}
\end{Shaded}

\begin{verbatim}
## [1] "x is not greater than 5"
\end{verbatim}

The \texttt{if} function is useful, but it can only ask one question at
a time. If you supply it with a vector of length greater than 1, it will
give a warning:

\begin{Shaded}
\begin{Highlighting}[]
\CommentTok{# vector from -5 to 5, excluding zero}
\NormalTok{xs =}\StringTok{ }\KeywordTok{c}\NormalTok{(}\OperatorTok{-}\DecValTok{5}\OperatorTok{:-}\DecValTok{1}\NormalTok{, }\DecValTok{1}\OperatorTok{:}\DecValTok{5}\NormalTok{)}

\CommentTok{# attempt a logical decision}
\ControlFlowTok{if}\NormalTok{ (xs }\OperatorTok{<}\StringTok{ }\DecValTok{0}\NormalTok{) }\KeywordTok{print}\NormalTok{(}\StringTok{"negative"}\NormalTok{)}
\end{Highlighting}
\end{Shaded}

\begin{verbatim}
## Warning in if (xs < 0) print("negative"): the condition has length > 1 and
## only the first element will be used
\end{verbatim}

\begin{verbatim}
## [1] "negative"
\end{verbatim}

Warnings are different than errors in that something still happens, but
it tells you that it might not be what you wanted, whereas an error
stops R altogether. In short, this warning is telling you that you
passed \texttt{if} a logical vector with more than 1 element, and that
it can only use one element so it's picking the first one. Because the
first element of \texttt{xs} is -5, \texttt{xs\ \textless{}\ 0}
evaluated to \texttt{TRUE}, and we got a \texttt{"negative"} printed
along with our warning.

To ask multiple questions at once, we must use \texttt{ifelse}. This
function is similar, but it combines the \texttt{if} and \texttt{else}
syntax into one useful function function:

\begin{Shaded}
\begin{Highlighting}[]
\KeywordTok{ifelse}\NormalTok{(xs }\OperatorTok{>}\StringTok{ }\DecValTok{0}\NormalTok{, }\StringTok{"positive"}\NormalTok{, }\StringTok{"negative"}\NormalTok{)}
\end{Highlighting}
\end{Shaded}

\begin{verbatim}
##  [1] "negative" "negative" "negative" "negative" "negative" "positive"
##  [7] "positive" "positive" "positive" "positive"
\end{verbatim}

The syntax is \texttt{ifelse(condition,\ do\_if\_TRUE,\ do\_if\_FALSE)}.
You can cbind the output with \texttt{xs} to verify it worked:

\begin{Shaded}
\begin{Highlighting}[]
\KeywordTok{cbind}\NormalTok{(}
\NormalTok{  xs,}
  \KeywordTok{ifelse}\NormalTok{(xs }\OperatorTok{>}\StringTok{ }\DecValTok{0}\NormalTok{, }\StringTok{"positive"}\NormalTok{, }\StringTok{"negative"}\NormalTok{)}
\NormalTok{)}
\end{Highlighting}
\end{Shaded}

\begin{verbatim}
##       xs             
##  [1,] "-5" "negative"
##  [2,] "-4" "negative"
##  [3,] "-3" "negative"
##  [4,] "-2" "negative"
##  [5,] "-1" "negative"
##  [6,] "1"  "positive"
##  [7,] "2"  "positive"
##  [8,] "3"  "positive"
##  [9,] "4"  "positive"
## [10,] "5"  "positive"
\end{verbatim}

Use \texttt{ifelse} to create a new variable in \texttt{dat} that
indicateds whether stream is big or small depending \textbf{whether
stream with is greater or less than 50}:

\begin{Shaded}
\begin{Highlighting}[]
\NormalTok{dat}\OperatorTok{$}\NormalTok{size_cat =}\StringTok{ }\KeywordTok{ifelse}\NormalTok{(dat}\OperatorTok{$}\NormalTok{stream_width }\OperatorTok{>}\StringTok{ }\DecValTok{50}\NormalTok{, }\StringTok{"big"}\NormalTok{, }\StringTok{"small"}\NormalTok{); }\KeywordTok{head}\NormalTok{(dat)}
\end{Highlighting}
\end{Shaded}

\begin{verbatim}
##     state stream_width   flow size_cat
## 1 Alabama        81.68 120.48      big
## 2 Alabama        57.76  85.90      big
## 3 Alabama        48.32  73.38    small
## 4 Alabama        31.63  46.55    small
## 5 Alabama        48.93  80.91    small
## 6 Georgia        39.42  57.63    small
\end{verbatim}

This says ``make a new variable in the data frame \texttt{dat} called
\texttt{size\_cat} and assign each row a `big' if \texttt{stream\_width}
is greater than 50 and a `small' if less than 50''.

One neat thing about \texttt{ifelse} is that you can nest multiple
statements inside another\footnote{You can nest \textbf{ALL} R
  functions, by the way.}. \textbf{What if we wanted three categories:
`small', `medium', and `large'?}

\begin{Shaded}
\begin{Highlighting}[]
\NormalTok{dat}\OperatorTok{$}\NormalTok{size_cat_fine =}\StringTok{ }\KeywordTok{ifelse}\NormalTok{(dat}\OperatorTok{$}\NormalTok{stream_width }\OperatorTok{<=}\StringTok{ }\DecValTok{40}\NormalTok{, }\StringTok{"small"}\NormalTok{,}
                           \KeywordTok{ifelse}\NormalTok{(dat}\OperatorTok{$}\NormalTok{stream_width }\OperatorTok{>}\StringTok{ }\DecValTok{40} \OperatorTok{&}\StringTok{ }\NormalTok{dat}\OperatorTok{$}\NormalTok{stream_width }\OperatorTok{<=}\StringTok{ }\DecValTok{70}\NormalTok{, }\StringTok{"medium"}\NormalTok{, }\StringTok{"big"}\NormalTok{)); }\KeywordTok{head}\NormalTok{(dat)}
\end{Highlighting}
\end{Shaded}

\begin{verbatim}
##     state stream_width   flow size_cat size_cat_fine
## 1 Alabama        81.68 120.48      big           big
## 2 Alabama        57.76  85.90      big        medium
## 3 Alabama        48.32  73.38    small        medium
## 4 Alabama        31.63  46.55    small         small
## 5 Alabama        48.93  80.91    small        medium
## 6 Georgia        39.42  57.63    small         small
\end{verbatim}

If the first condition is \texttt{TRUE}, then it will give that row a
``small''. If not, it will start another \texttt{ifelse} to ask if the
\texttt{stream\_width} is greater than 40 \emph{and} less than or equal
to 70. If so, it will give it a ``medium'', if not it will get a
``big''. This is confusing at first. Not all nesting or embedding of
functions are this complex, but this is a neat example. Without
\texttt{ifelse}, you would have to use as many \texttt{if} statements as
there are elements in \texttt{dat\$stream\_width}.

\section{Writing Output Files}\label{writing-output-files}

\subsection{.csv Files}\label{csv-files}

Now that you have made some new variables in your data frame, you may
want to save this work in the form of a new \texttt{.csv} file. To do
this, you can use the \texttt{write.csv} function:

\begin{Shaded}
\begin{Highlighting}[]
\KeywordTok{write.csv}\NormalTok{(dat, }\StringTok{"updated_streams.csv"}\NormalTok{, }\DataTypeTok{row.names =}\NormalTok{ F)}
\end{Highlighting}
\end{Shaded}

The first argument is the data frame (or matrix) to write, the second is
what you want to call it (don't forget the \texttt{.csv}!), and
\texttt{row.names\ =\ F} tells R to not include the row names (because
they are just numbers in this case). R puts the file in your working
directory unless you tell it otherwise. To put it somewhere else, type
in the path with the new file name at the end.

\subsection{Saving R Objects}\label{saving-r-objects}

Another method is to save R objects directly to the working directory.
You can save the new data frame using:

\begin{Shaded}
\begin{Highlighting}[]
\KeywordTok{save}\NormalTok{(dat, }\DataTypeTok{file =} \StringTok{"updated_streams"}\NormalTok{)}
\end{Highlighting}
\end{Shaded}

Then try removing the \texttt{dat} object from your current session
(\texttt{rm(dat)}) and loading it back in using:

\begin{Shaded}
\begin{Highlighting}[]
\KeywordTok{rm}\NormalTok{(dat); }\KeywordTok{head}\NormalTok{(dat)  }\CommentTok{# should give error}
\KeywordTok{load}\NormalTok{(}\DataTypeTok{file =} \StringTok{"updated_streams"}\NormalTok{)}
\KeywordTok{head}\NormalTok{(dat) }\CommentTok{# should show first 6 rows}
\end{Highlighting}
\end{Shaded}

\section{User-Defined Functions}\label{user-defined-functions}

Sometimes you may want R to carry out a specific task, but there is no
built-in function to do it. In this case, you can write your own
functions. This one of the parts of R that makes it incredibly flexible,
though you will only get a small taste of this topic here. We will be
going into more detail in later Chapters, particularly in Chapter
\ref{ch5}.

First, you must think of a name for your function (e.g.,
\texttt{myfun}). Then, you specify that you want the object
\texttt{myfun} to be a function by using using the \texttt{function}
function (I know). Then, in parentheses, you specify any arguments that
you want to use within the function to carry out the specific task. Open
and closed curly braces specify the start and end of you function body,
i.e., the code that specifies how it uses the arguments to do its job.

Here's the general syntax for specifying your own function:

\begin{Shaded}
\begin{Highlighting}[]
\NormalTok{myfun =}\StringTok{ }\ControlFlowTok{function}\NormalTok{(arg1) \{}
  \CommentTok{# function body goes here}
  \CommentTok{# use arg1 to do something}
  
  \CommentTok{# return something as last step}
\NormalTok{\}}
\end{Highlighting}
\end{Shaded}

Write a general function to take any number \texttt{x} to any power
\texttt{y}:

\begin{Shaded}
\begin{Highlighting}[]
\NormalTok{power =}\StringTok{ }\ControlFlowTok{function}\NormalTok{(x, y)\{}
\NormalTok{  x}\OperatorTok{^}\NormalTok{y}
\NormalTok{\}}
\end{Highlighting}
\end{Shaded}

After typing and running the function code (the function is an object
that must be assigned), try using it:

\begin{Shaded}
\begin{Highlighting}[]
\KeywordTok{power}\NormalTok{(}\DataTypeTok{x =} \DecValTok{5}\NormalTok{, }\DataTypeTok{y =} \DecValTok{3}\NormalTok{)}
\end{Highlighting}
\end{Shaded}

\begin{verbatim}
## [1] 125
\end{verbatim}

Remember, you can nest or embed functions:

\begin{Shaded}
\begin{Highlighting}[]
\KeywordTok{power}\NormalTok{(}\KeywordTok{power}\NormalTok{(}\DecValTok{5}\NormalTok{,}\DecValTok{2}\NormalTok{),}\DecValTok{2}\NormalTok{)}
\end{Highlighting}
\end{Shaded}

\begin{verbatim}
## [1] 625
\end{verbatim}

This is the equivalent of \((5^2)^2\).

\section*{EXERCISE 1B}\label{exercise-1b}
\addcontentsline{toc}{section}{EXERCISE 1B}

These data are from a hypothetical pond experiment where you added
nutrients to some mesocosoms and counted the densities of 4 different
zooplankton taxa. In this experiment, there were two ponds, two
treatments per pond, and five replicates of each treatment. There is one
error in the dataset ``ponds.csv'' (remember, found in this repository:
\url{https://github.com/bstaton1/au-r-workshop-data/tree/master}). After
you download the data and place it in the appropriate directory, make
sure you open this file and fix it \emph{before} you bring it into R.
Refer back to the information about reading in data (Section \ref{read})
to make sure you find the error.

\begin{enumerate}
\def\labelenumi{\arabic{enumi}.}
\item
  Read in the data to R and assign it to an object.
\item
  Calculate some basic summary statistics of your data using the
  \texttt{summary} function.
\item
  Calculate the mean chlorophyll \emph{a} for each pond (\emph{Hint:
  pond is a grouping variable})
\item
  Calculate the mean number of \emph{Chaoborus} for each treatment in
  each pond using \texttt{tapply}. (\emph{Hint: You can group by two
  variables with:}
  \texttt{tapply(dat\$var,\ list(dat\$grp1,\ dat\$grp2),\ fun)}.
\item
  Use the more general \texttt{apply} function to calculate the variance
  for each zooplankton taxa found only in pond S-28.
\item
  Create a new variable called \texttt{prod} in the data frame that
  represents the quantity of chlorophyll \emph{a} in each replicate. If
  the chlorophyll \emph{a} in the replicate is greater than 30 give it a
  ``high'', otherwise give it a ``low''. (\emph{Hint: are you asking a
  question of a single number or multiple numbers? How should this
  change the strategy you use?})
\end{enumerate}

\subsection*{EXERCISE 1B BONUS}\label{exercise-1b-bonus}
\addcontentsline{toc}{subsection}{EXERCISE 1B BONUS}

\begin{enumerate}
\def\labelenumi{\arabic{enumi}.}
\item
  Create a new function called \texttt{product} that multiplies two
  numbers that you specify.
\item
  Modify your function to print a message to the console and return the
  value \texttt{if} it meets a condition and to print another message
  and not return the value if it doesn't.
\end{enumerate}

\chapter{Base R Plotting Basics}\label{ch2}

\chapter{Basic Statistics}\label{ch3}

\chapter{Simulation and Randomization}\label{ch4}

\chapter{Large Data Manipulation}\label{ch5}

\bibliography{book.bib,packages.bib}


\end{document}
